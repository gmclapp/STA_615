\documentclass[]{article}
\usepackage{lmodern}
\usepackage{amssymb,amsmath}
\usepackage{ifxetex,ifluatex}
\usepackage{fixltx2e} % provides \textsubscript
\ifnum 0\ifxetex 1\fi\ifluatex 1\fi=0 % if pdftex
  \usepackage[T1]{fontenc}
  \usepackage[utf8]{inputenc}
\else % if luatex or xelatex
  \ifxetex
    \usepackage{mathspec}
  \else
    \usepackage{fontspec}
  \fi
  \defaultfontfeatures{Ligatures=TeX,Scale=MatchLowercase}
\fi
% use upquote if available, for straight quotes in verbatim environments
\IfFileExists{upquote.sty}{\usepackage{upquote}}{}
% use microtype if available
\IfFileExists{microtype.sty}{%
\usepackage{microtype}
\UseMicrotypeSet[protrusion]{basicmath} % disable protrusion for tt fonts
}{}
\usepackage[margin=1in]{geometry}
\usepackage{hyperref}
\hypersetup{unicode=true,
            pdftitle={Probability Work},
            pdfauthor={D. Zeitler},
            pdfborder={0 0 0},
            breaklinks=true}
\urlstyle{same}  % don't use monospace font for urls
\usepackage{color}
\usepackage{fancyvrb}
\newcommand{\VerbBar}{|}
\newcommand{\VERB}{\Verb[commandchars=\\\{\}]}
\DefineVerbatimEnvironment{Highlighting}{Verbatim}{commandchars=\\\{\}}
% Add ',fontsize=\small' for more characters per line
\usepackage{framed}
\definecolor{shadecolor}{RGB}{248,248,248}
\newenvironment{Shaded}{\begin{snugshade}}{\end{snugshade}}
\newcommand{\KeywordTok}[1]{\textcolor[rgb]{0.13,0.29,0.53}{\textbf{#1}}}
\newcommand{\DataTypeTok}[1]{\textcolor[rgb]{0.13,0.29,0.53}{#1}}
\newcommand{\DecValTok}[1]{\textcolor[rgb]{0.00,0.00,0.81}{#1}}
\newcommand{\BaseNTok}[1]{\textcolor[rgb]{0.00,0.00,0.81}{#1}}
\newcommand{\FloatTok}[1]{\textcolor[rgb]{0.00,0.00,0.81}{#1}}
\newcommand{\ConstantTok}[1]{\textcolor[rgb]{0.00,0.00,0.00}{#1}}
\newcommand{\CharTok}[1]{\textcolor[rgb]{0.31,0.60,0.02}{#1}}
\newcommand{\SpecialCharTok}[1]{\textcolor[rgb]{0.00,0.00,0.00}{#1}}
\newcommand{\StringTok}[1]{\textcolor[rgb]{0.31,0.60,0.02}{#1}}
\newcommand{\VerbatimStringTok}[1]{\textcolor[rgb]{0.31,0.60,0.02}{#1}}
\newcommand{\SpecialStringTok}[1]{\textcolor[rgb]{0.31,0.60,0.02}{#1}}
\newcommand{\ImportTok}[1]{#1}
\newcommand{\CommentTok}[1]{\textcolor[rgb]{0.56,0.35,0.01}{\textit{#1}}}
\newcommand{\DocumentationTok}[1]{\textcolor[rgb]{0.56,0.35,0.01}{\textbf{\textit{#1}}}}
\newcommand{\AnnotationTok}[1]{\textcolor[rgb]{0.56,0.35,0.01}{\textbf{\textit{#1}}}}
\newcommand{\CommentVarTok}[1]{\textcolor[rgb]{0.56,0.35,0.01}{\textbf{\textit{#1}}}}
\newcommand{\OtherTok}[1]{\textcolor[rgb]{0.56,0.35,0.01}{#1}}
\newcommand{\FunctionTok}[1]{\textcolor[rgb]{0.00,0.00,0.00}{#1}}
\newcommand{\VariableTok}[1]{\textcolor[rgb]{0.00,0.00,0.00}{#1}}
\newcommand{\ControlFlowTok}[1]{\textcolor[rgb]{0.13,0.29,0.53}{\textbf{#1}}}
\newcommand{\OperatorTok}[1]{\textcolor[rgb]{0.81,0.36,0.00}{\textbf{#1}}}
\newcommand{\BuiltInTok}[1]{#1}
\newcommand{\ExtensionTok}[1]{#1}
\newcommand{\PreprocessorTok}[1]{\textcolor[rgb]{0.56,0.35,0.01}{\textit{#1}}}
\newcommand{\AttributeTok}[1]{\textcolor[rgb]{0.77,0.63,0.00}{#1}}
\newcommand{\RegionMarkerTok}[1]{#1}
\newcommand{\InformationTok}[1]{\textcolor[rgb]{0.56,0.35,0.01}{\textbf{\textit{#1}}}}
\newcommand{\WarningTok}[1]{\textcolor[rgb]{0.56,0.35,0.01}{\textbf{\textit{#1}}}}
\newcommand{\AlertTok}[1]{\textcolor[rgb]{0.94,0.16,0.16}{#1}}
\newcommand{\ErrorTok}[1]{\textcolor[rgb]{0.64,0.00,0.00}{\textbf{#1}}}
\newcommand{\NormalTok}[1]{#1}
\usepackage{graphicx,grffile}
\makeatletter
\def\maxwidth{\ifdim\Gin@nat@width>\linewidth\linewidth\else\Gin@nat@width\fi}
\def\maxheight{\ifdim\Gin@nat@height>\textheight\textheight\else\Gin@nat@height\fi}
\makeatother
% Scale images if necessary, so that they will not overflow the page
% margins by default, and it is still possible to overwrite the defaults
% using explicit options in \includegraphics[width, height, ...]{}
\setkeys{Gin}{width=\maxwidth,height=\maxheight,keepaspectratio}
\IfFileExists{parskip.sty}{%
\usepackage{parskip}
}{% else
\setlength{\parindent}{0pt}
\setlength{\parskip}{6pt plus 2pt minus 1pt}
}
\setlength{\emergencystretch}{3em}  % prevent overfull lines
\providecommand{\tightlist}{%
  \setlength{\itemsep}{0pt}\setlength{\parskip}{0pt}}
\setcounter{secnumdepth}{0}
% Redefines (sub)paragraphs to behave more like sections
\ifx\paragraph\undefined\else
\let\oldparagraph\paragraph
\renewcommand{\paragraph}[1]{\oldparagraph{#1}\mbox{}}
\fi
\ifx\subparagraph\undefined\else
\let\oldsubparagraph\subparagraph
\renewcommand{\subparagraph}[1]{\oldsubparagraph{#1}\mbox{}}
\fi

%%% Use protect on footnotes to avoid problems with footnotes in titles
\let\rmarkdownfootnote\footnote%
\def\footnote{\protect\rmarkdownfootnote}

%%% Change title format to be more compact
\usepackage{titling}

% Create subtitle command for use in maketitle
\newcommand{\subtitle}[1]{
  \posttitle{
    \begin{center}\large#1\end{center}
    }
}

\setlength{\droptitle}{-2em}

  \title{Probability Work}
    \pretitle{\vspace{\droptitle}\centering\huge}
  \posttitle{\par}
    \author{D. Zeitler}
    \preauthor{\centering\large\emph}
  \postauthor{\par}
      \predate{\centering\large\emph}
  \postdate{\par}
    \date{September 7, 2017}


\begin{document}
\maketitle

We're just going to explore probabilities and random numbers with R and
mosaic.

\begin{Shaded}
\begin{Highlighting}[]
\CommentTok{#library(tidyverse)}
\KeywordTok{library}\NormalTok{(mosaic)}
\CommentTok{#library(ggformula)}
\end{Highlighting}
\end{Shaded}

R provides several probability distributions with associated functions.

\begin{itemize}
\tightlist
\item
  d: density or probability
\item
  q: quantile
\item
  p: cumulative probability
\item
  r: sample random values
\end{itemize}

Along with the extensive probability calculations provided in R, the
mosaic package provides a couple of useful functions.

\emph{xp: cumulative probability (right and left tail) with graphic
}plot: plots pdf/pmf or cdf

\section{First we will go through basic R probability, then mosaic
stuff.}\label{first-we-will-go-through-basic-r-probability-then-mosaic-stuff.}

Getting the probability for Z between -1.96 and 1.96 is easy:

Subtracting the probability in the tails from one.

\begin{Shaded}
\begin{Highlighting}[]
\DecValTok{1}\OperatorTok{-}\KeywordTok{pnorm}\NormalTok{(}\OperatorTok{-}\FloatTok{1.96}\NormalTok{)}\OperatorTok{*}\DecValTok{2}
\end{Highlighting}
\end{Shaded}

\begin{verbatim}
## [1] 0.9500042
\end{verbatim}

Using the definition, subtract the cdf at the lower limit from the cdf
at the upper limit.

\begin{Shaded}
\begin{Highlighting}[]
\KeywordTok{pnorm}\NormalTok{(}\FloatTok{1.96}\NormalTok{) }\OperatorTok{-}\StringTok{ }\KeywordTok{pnorm}\NormalTok{(}\OperatorTok{-}\FloatTok{1.96}\NormalTok{)}
\end{Highlighting}
\end{Shaded}

\begin{verbatim}
## [1] 0.9500042
\end{verbatim}

\subsection{Reverse probabilities}\label{reverse-probabilities}

We used 1.96, that's an approximate 95\% value. Get the z-value that
gives us exactly 0.95, not just close.

\begin{Shaded}
\begin{Highlighting}[]
\KeywordTok{qnorm}\NormalTok{(.}\DecValTok{025}\NormalTok{)}
\end{Highlighting}
\end{Shaded}

\begin{verbatim}
## [1] -1.959964
\end{verbatim}

Or if we don't want to think about removing that negative sign\ldots{}

\begin{Shaded}
\begin{Highlighting}[]
\KeywordTok{qnorm}\NormalTok{(}\DecValTok{1}\OperatorTok{-}\NormalTok{.}\DecValTok{025}\NormalTok{)}
\end{Highlighting}
\end{Shaded}

\begin{verbatim}
## [1] 1.959964
\end{verbatim}

These are all using left tail probabilities. We can easily do right tail
probabiliteis.

\begin{Shaded}
\begin{Highlighting}[]
\KeywordTok{pnorm}\NormalTok{(}\FloatTok{1.96}\NormalTok{,}\DataTypeTok{lower.tail =} \OtherTok{FALSE}\NormalTok{)}
\end{Highlighting}
\end{Shaded}

\begin{verbatim}
## [1] 0.0249979
\end{verbatim}

\begin{Shaded}
\begin{Highlighting}[]
\KeywordTok{qnorm}\NormalTok{(.}\DecValTok{025}\NormalTok{,}\DataTypeTok{lower.tail =} \OtherTok{FALSE}\NormalTok{)}
\end{Highlighting}
\end{Shaded}

\begin{verbatim}
## [1] 1.959964
\end{verbatim}

Another neat trick is to put the value right into my text, say here:
1.959964.

\section{Maybe some graphics and
summaries}\label{maybe-some-graphics-and-summaries}

\begin{verbatim}
##      speed           dist       
##  Min.   : 4.0   Min.   :  2.00  
##  1st Qu.:12.0   1st Qu.: 26.00  
##  Median :15.0   Median : 36.00  
##  Mean   :15.4   Mean   : 42.98  
##  3rd Qu.:19.0   3rd Qu.: 56.00  
##  Max.   :25.0   Max.   :120.00
\end{verbatim}

\includegraphics{ProbabilityMusings_files/figure-latex/unnamed-chunk-6-1.pdf}

\section{Look at speed}\label{look-at-speed}

\begin{Shaded}
\begin{Highlighting}[]
\KeywordTok{histogram}\NormalTok{(}\OperatorTok{~}\NormalTok{speed,}\DataTypeTok{data=}\NormalTok{cars)}
\end{Highlighting}
\end{Shaded}

\includegraphics{ProbabilityMusings_files/figure-latex/unnamed-chunk-7-1.pdf}

\begin{Shaded}
\begin{Highlighting}[]
\KeywordTok{histogram}\NormalTok{(}\OperatorTok{~}\NormalTok{dist,}\DataTypeTok{data=}\NormalTok{cars)}
\end{Highlighting}
\end{Shaded}

\includegraphics{ProbabilityMusings_files/figure-latex/unnamed-chunk-8-1.pdf}

Although we can assess normality by looking at the histogram, a better
way is to use a quantile plot.

\begin{Shaded}
\begin{Highlighting}[]
\KeywordTok{gf_qq}\NormalTok{(}\OperatorTok{~}\NormalTok{speed,}\DataTypeTok{data=}\NormalTok{cars) }\OperatorTok
\StringTok{  }\KeywordTok{gf_qqline}\NormalTok{(}\OperatorTok{~}\NormalTok{speed,}\DataTypeTok{data=}\NormalTok{cars)}
\end{Highlighting}
\end{Shaded}

\includegraphics{ProbabilityMusings_files/figure-latex/unnamed-chunk-9-1.pdf}

\begin{Shaded}
\begin{Highlighting}[]
\KeywordTok{gf_qq}\NormalTok{(}\OperatorTok{~}\NormalTok{dist,}\DataTypeTok{data=}\NormalTok{cars) }\OperatorTok
\StringTok{  }\KeywordTok{gf_qqline}\NormalTok{(}\OperatorTok{~}\NormalTok{dist,}\DataTypeTok{data=}\NormalTok{cars)}
\end{Highlighting}
\end{Shaded}

\includegraphics{ProbabilityMusings_files/figure-latex/unnamed-chunk-10-1.pdf}

\section{further probability stuff}\label{further-probability-stuff}

Let's say we're making widgets. They're produced with a mean diameter 34
and standard deviation of 3.

What is the probability of getting widgets that are smaller than 24.

\begin{Shaded}
\begin{Highlighting}[]
\KeywordTok{pnorm}\NormalTok{(}\DecValTok{24}\NormalTok{,}\DataTypeTok{lower.tail =}\NormalTok{ T,}\DataTypeTok{mean =} \DecValTok{34}\NormalTok{, }\DataTypeTok{sd =} \DecValTok{3}\NormalTok{)}
\end{Highlighting}
\end{Shaded}

\begin{verbatim}
## [1] 0.0004290603
\end{verbatim}

\section{integrating}\label{integrating}

Probability Z \textless{} 0.

\begin{Shaded}
\begin{Highlighting}[]
\KeywordTok{integrate}\NormalTok{(dnorm, }\OperatorTok{-}\OtherTok{Inf}\NormalTok{, }\DecValTok{0}\NormalTok{)}
\end{Highlighting}
\end{Shaded}

\begin{verbatim}
## 0.5 with absolute error < 4.7e-05
\end{verbatim}

\section{mosaic functions}\label{mosaic-functions}

\subsection{xpnorm}\label{xpnorm}

\begin{Shaded}
\begin{Highlighting}[]
\KeywordTok{xpnorm}\NormalTok{(}\FloatTok{1.96}\NormalTok{, }\DataTypeTok{mean=}\DecValTok{0}\NormalTok{, }\DataTypeTok{sd=}\DecValTok{1}\NormalTok{)}
\end{Highlighting}
\end{Shaded}

\begin{verbatim}
## 
\end{verbatim}

\begin{verbatim}
## If X ~ N(0, 1), then
\end{verbatim}

\begin{verbatim}
##  P(X <= 1.96) = P(Z <= 1.96) = 0.975
\end{verbatim}

\begin{verbatim}
##  P(X >  1.96) = P(Z >  1.96) = 0.025
\end{verbatim}

\begin{verbatim}
## 
\end{verbatim}

\includegraphics{ProbabilityMusings_files/figure-latex/unnamed-chunk-13-1.pdf}

\begin{verbatim}
## [1] 0.9750021
\end{verbatim}

\subsection{plotDist}\label{plotdist}

\begin{Shaded}
\begin{Highlighting}[]
\KeywordTok{plotDist}\NormalTok{(}\StringTok{'norm'}\NormalTok{, }\DataTypeTok{mean=}\DecValTok{0}\NormalTok{, }\DataTypeTok{sd=}\DecValTok{1}\NormalTok{, }\DataTypeTok{kind=}\StringTok{'cdf'}\NormalTok{,}\DataTypeTok{main=}\StringTok{'cdf'}\NormalTok{)}
\end{Highlighting}
\end{Shaded}

\includegraphics{ProbabilityMusings_files/figure-latex/unnamed-chunk-14-1.pdf}

\begin{Shaded}
\begin{Highlighting}[]
\KeywordTok{plotDist}\NormalTok{(}\StringTok{'norm'}\NormalTok{, }\DataTypeTok{mean=}\DecValTok{0}\NormalTok{, }\DataTypeTok{sd=}\DecValTok{1}\NormalTok{, }\DataTypeTok{kind=}\StringTok{'qq'}\NormalTok{,}\DataTypeTok{main=}\StringTok{'q-q'}\NormalTok{)}
\end{Highlighting}
\end{Shaded}

\includegraphics{ProbabilityMusings_files/figure-latex/unnamed-chunk-14-2.pdf}

\begin{Shaded}
\begin{Highlighting}[]
\KeywordTok{plotDist}\NormalTok{(}\StringTok{'norm'}\NormalTok{, }\DataTypeTok{mean=}\DecValTok{0}\NormalTok{, }\DataTypeTok{sd=}\DecValTok{1}\NormalTok{, }\DataTypeTok{kind=}\StringTok{'density'}\NormalTok{,}\DataTypeTok{main=}\StringTok{'density'}\NormalTok{)}
\end{Highlighting}
\end{Shaded}

\includegraphics{ProbabilityMusings_files/figure-latex/unnamed-chunk-14-3.pdf}

\begin{Shaded}
\begin{Highlighting}[]
\KeywordTok{plotDist}\NormalTok{(}\StringTok{'norm'}\NormalTok{, }\DataTypeTok{mean=}\DecValTok{0}\NormalTok{, }\DataTypeTok{sd=}\DecValTok{1}\NormalTok{, }\DataTypeTok{kind=}\StringTok{'histogram'}\NormalTok{,}\DataTypeTok{main=}\StringTok{'histogram'}\NormalTok{)}
\end{Highlighting}
\end{Shaded}

\includegraphics{ProbabilityMusings_files/figure-latex/unnamed-chunk-14-4.pdf}
plotDist can be used for discrete too.

\begin{Shaded}
\begin{Highlighting}[]
\KeywordTok{plotDist}\NormalTok{(}\StringTok{'binom'}\NormalTok{, }\DataTypeTok{size=}\DecValTok{25}\NormalTok{, }\DataTypeTok{prob=}\NormalTok{.}\DecValTok{25}\NormalTok{, }\DataTypeTok{kind=}\StringTok{'cdf'}\NormalTok{,}\DataTypeTok{main=}\StringTok{'cdf'}\NormalTok{)}
\end{Highlighting}
\end{Shaded}

\includegraphics{ProbabilityMusings_files/figure-latex/unnamed-chunk-15-1.pdf}

\begin{Shaded}
\begin{Highlighting}[]
\KeywordTok{plotDist}\NormalTok{(}\StringTok{'binom'}\NormalTok{, }\DataTypeTok{size=}\DecValTok{25}\NormalTok{, }\DataTypeTok{prob=}\NormalTok{.}\DecValTok{25}\NormalTok{, }\DataTypeTok{kind=}\StringTok{'qq'}\NormalTok{,}\DataTypeTok{main=}\StringTok{'q-q'}\NormalTok{)}
\end{Highlighting}
\end{Shaded}

\includegraphics{ProbabilityMusings_files/figure-latex/unnamed-chunk-15-2.pdf}

\begin{Shaded}
\begin{Highlighting}[]
\KeywordTok{plotDist}\NormalTok{(}\StringTok{'binom'}\NormalTok{, }\DataTypeTok{size=}\DecValTok{25}\NormalTok{, }\DataTypeTok{prob=}\NormalTok{.}\DecValTok{25}\NormalTok{, }\DataTypeTok{kind=}\StringTok{'density'}\NormalTok{,}\DataTypeTok{main=}\StringTok{'density'}\NormalTok{)}
\end{Highlighting}
\end{Shaded}

\includegraphics{ProbabilityMusings_files/figure-latex/unnamed-chunk-15-3.pdf}

\begin{Shaded}
\begin{Highlighting}[]
\KeywordTok{plotDist}\NormalTok{(}\StringTok{'binom'}\NormalTok{, }\DataTypeTok{size=}\DecValTok{25}\NormalTok{, }\DataTypeTok{prob=}\NormalTok{.}\DecValTok{25}\NormalTok{, }\DataTypeTok{kind=}\StringTok{'histogram'}\NormalTok{,}\DataTypeTok{main=}\StringTok{'histogram'}\NormalTok{)}
\end{Highlighting}
\end{Shaded}

\includegraphics{ProbabilityMusings_files/figure-latex/unnamed-chunk-15-4.pdf}

Overlaying things:

\begin{Shaded}
\begin{Highlighting}[]
\KeywordTok{plotDist}\NormalTok{(}\StringTok{"binom"}\NormalTok{, }\DataTypeTok{size=}\DecValTok{100}\NormalTok{, }\DataTypeTok{prob=}\NormalTok{.}\DecValTok{3}\NormalTok{, }\DataTypeTok{col=}\StringTok{'black'}\NormalTok{, }\DataTypeTok{lwd=}\DecValTok{3}\NormalTok{, }\DataTypeTok{pch=}\DecValTok{16}\NormalTok{)}
\end{Highlighting}
\end{Shaded}

\includegraphics{ProbabilityMusings_files/figure-latex/unnamed-chunk-16-1.pdf}

\begin{Shaded}
\begin{Highlighting}[]
\KeywordTok{plotDist}\NormalTok{(}\StringTok{"norm"}\NormalTok{, }\DataTypeTok{mean=}\DecValTok{30}\NormalTok{, }\DataTypeTok{sd=}\KeywordTok{sqrt}\NormalTok{(}\DecValTok{100}\OperatorTok{*}\NormalTok{.}\DecValTok{3}\OperatorTok{*}\NormalTok{.}\DecValTok{7}\NormalTok{), }\DataTypeTok{groups=}\KeywordTok{abs}\NormalTok{(x}\OperatorTok{-}\DecValTok{30}\NormalTok{)}\OperatorTok{>}\DecValTok{6}\NormalTok{, }\DataTypeTok{type=}\StringTok{"h"}\NormalTok{, }\DataTypeTok{under=}\OtherTok{TRUE}\NormalTok{)}
\end{Highlighting}
\end{Shaded}

\includegraphics{ProbabilityMusings_files/figure-latex/unnamed-chunk-16-2.pdf}

Using fitdistr finds the distribution parameters based on a set of data.
For example we generate 1000 standard normals and try fitting it to the
normal distribution. The output is the estimated mean and standard
deviation with their standard deviations.

\begin{Shaded}
\begin{Highlighting}[]
\KeywordTok{fitdistr}\NormalTok{(}\KeywordTok{rnorm}\NormalTok{(}\DecValTok{1000}\NormalTok{),}\DataTypeTok{densfun =} \StringTok{'normal'}\NormalTok{)}
\end{Highlighting}
\end{Shaded}

\begin{verbatim}
##       mean           sd     
##   -0.01606216    0.97564721 
##  ( 0.03085267) ( 0.02181613)
\end{verbatim}


\end{document}
